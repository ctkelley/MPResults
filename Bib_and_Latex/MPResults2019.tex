
    




    
\documentclass[11pt]{article}

    
    \usepackage[breakable]{tcolorbox}
    \tcbset{nobeforeafter} % prevents tcolorboxes being placing in paragraphs
    \usepackage{float}
    \floatplacement{figure}{H} % forces figures to be placed at the correct location
    
    \usepackage[T1]{fontenc}
    % Nicer default font (+ math font) than Computer Modern for most use cases
    \usepackage{mathpazo}

    % Basic figure setup, for now with no caption control since it's done
    % automatically by Pandoc (which extracts ![](path) syntax from Markdown).
    \usepackage{graphicx}
    % We will generate all images so they have a width \maxwidth. This means
    % that they will get their normal width if they fit onto the page, but
    % are scaled down if they would overflow the margins.
    \makeatletter
    \def\maxwidth{\ifdim\Gin@nat@width>\linewidth\linewidth
    \else\Gin@nat@width\fi}
    \makeatother
    \let\Oldincludegraphics\includegraphics
    % Set max figure width to be 80% of text width, for now hardcoded.
    \renewcommand{\includegraphics}[1]{\Oldincludegraphics[width=.8\maxwidth]{#1}}
    % Ensure that by default, figures have no caption (until we provide a
    % proper Figure object with a Caption API and a way to capture that
    % in the conversion process - todo).
    \usepackage{caption}
    \DeclareCaptionLabelFormat{nolabel}{}
    \captionsetup{labelformat=nolabel}

    \usepackage{adjustbox} % Used to constrain images to a maximum size 
    \usepackage{xcolor} % Allow colors to be defined
    \usepackage{enumerate} % Needed for markdown enumerations to work
    \usepackage{geometry} % Used to adjust the document margins
    \usepackage{amsmath} % Equations
    \usepackage{amssymb} % Equations
    \usepackage{textcomp} % defines textquotesingle
    % Hack from http://tex.stackexchange.com/a/47451/13684:
    \AtBeginDocument{%
        \def\PYZsq{\textquotesingle}% Upright quotes in Pygmentized code
    }
    \usepackage{upquote} % Upright quotes for verbatim code
    \usepackage{eurosym} % defines \euro
    \usepackage[mathletters]{ucs} % Extended unicode (utf-8) support
    \usepackage[utf8x]{inputenc} % Allow utf-8 characters in the tex document
    \usepackage{fancyvrb} % verbatim replacement that allows latex
    \usepackage{grffile} % extends the file name processing of package graphics 
                         % to support a larger range 
    % The hyperref package gives us a pdf with properly built
    % internal navigation ('pdf bookmarks' for the table of contents,
    % internal cross-reference links, web links for URLs, etc.)
    \usepackage{hyperref}
    \usepackage{longtable} % longtable support required by pandoc >1.10
    \usepackage{booktabs}  % table support for pandoc > 1.12.2
    \usepackage[inline]{enumitem} % IRkernel/repr support (it uses the enumerate* environment)
    \usepackage[normalem]{ulem} % ulem is needed to support strikethroughs (\sout)
                                % normalem makes italics be italics, not underlines
    \usepackage{mathrsfs}
    

    
    % Colors for the hyperref package
    \definecolor{urlcolor}{rgb}{0,.145,.698}
    \definecolor{linkcolor}{rgb}{.71,0.21,0.01}
    \definecolor{citecolor}{rgb}{.12,.54,.11}

    % ANSI colors
    \definecolor{ansi-black}{HTML}{3E424D}
    \definecolor{ansi-black-intense}{HTML}{282C36}
    \definecolor{ansi-red}{HTML}{E75C58}
    \definecolor{ansi-red-intense}{HTML}{B22B31}
    \definecolor{ansi-green}{HTML}{00A250}
    \definecolor{ansi-green-intense}{HTML}{007427}
    \definecolor{ansi-yellow}{HTML}{DDB62B}
    \definecolor{ansi-yellow-intense}{HTML}{B27D12}
    \definecolor{ansi-blue}{HTML}{208FFB}
    \definecolor{ansi-blue-intense}{HTML}{0065CA}
    \definecolor{ansi-magenta}{HTML}{D160C4}
    \definecolor{ansi-magenta-intense}{HTML}{A03196}
    \definecolor{ansi-cyan}{HTML}{60C6C8}
    \definecolor{ansi-cyan-intense}{HTML}{258F8F}
    \definecolor{ansi-white}{HTML}{C5C1B4}
    \definecolor{ansi-white-intense}{HTML}{A1A6B2}
    \definecolor{ansi-default-inverse-fg}{HTML}{FFFFFF}
    \definecolor{ansi-default-inverse-bg}{HTML}{000000}

    % commands and environments needed by pandoc snippets
    % extracted from the output of `pandoc -s`
    \providecommand{\tightlist}{%
      \setlength{\itemsep}{0pt}\setlength{\parskip}{0pt}}
    \DefineVerbatimEnvironment{Highlighting}{Verbatim}{commandchars=\\\{\}}
    % Add ',fontsize=\small' for more characters per line
    \newenvironment{Shaded}{}{}
    \newcommand{\KeywordTok}[1]{\textcolor[rgb]{0.00,0.44,0.13}{\textbf{{#1}}}}
    \newcommand{\DataTypeTok}[1]{\textcolor[rgb]{0.56,0.13,0.00}{{#1}}}
    \newcommand{\DecValTok}[1]{\textcolor[rgb]{0.25,0.63,0.44}{{#1}}}
    \newcommand{\BaseNTok}[1]{\textcolor[rgb]{0.25,0.63,0.44}{{#1}}}
    \newcommand{\FloatTok}[1]{\textcolor[rgb]{0.25,0.63,0.44}{{#1}}}
    \newcommand{\CharTok}[1]{\textcolor[rgb]{0.25,0.44,0.63}{{#1}}}
    \newcommand{\StringTok}[1]{\textcolor[rgb]{0.25,0.44,0.63}{{#1}}}
    \newcommand{\CommentTok}[1]{\textcolor[rgb]{0.38,0.63,0.69}{\textit{{#1}}}}
    \newcommand{\OtherTok}[1]{\textcolor[rgb]{0.00,0.44,0.13}{{#1}}}
    \newcommand{\AlertTok}[1]{\textcolor[rgb]{1.00,0.00,0.00}{\textbf{{#1}}}}
    \newcommand{\FunctionTok}[1]{\textcolor[rgb]{0.02,0.16,0.49}{{#1}}}
    \newcommand{\RegionMarkerTok}[1]{{#1}}
    \newcommand{\ErrorTok}[1]{\textcolor[rgb]{1.00,0.00,0.00}{\textbf{{#1}}}}
    \newcommand{\NormalTok}[1]{{#1}}
    
    % Additional commands for more recent versions of Pandoc
    \newcommand{\ConstantTok}[1]{\textcolor[rgb]{0.53,0.00,0.00}{{#1}}}
    \newcommand{\SpecialCharTok}[1]{\textcolor[rgb]{0.25,0.44,0.63}{{#1}}}
    \newcommand{\VerbatimStringTok}[1]{\textcolor[rgb]{0.25,0.44,0.63}{{#1}}}
    \newcommand{\SpecialStringTok}[1]{\textcolor[rgb]{0.73,0.40,0.53}{{#1}}}
    \newcommand{\ImportTok}[1]{{#1}}
    \newcommand{\DocumentationTok}[1]{\textcolor[rgb]{0.73,0.13,0.13}{\textit{{#1}}}}
    \newcommand{\AnnotationTok}[1]{\textcolor[rgb]{0.38,0.63,0.69}{\textbf{\textit{{#1}}}}}
    \newcommand{\CommentVarTok}[1]{\textcolor[rgb]{0.38,0.63,0.69}{\textbf{\textit{{#1}}}}}
    \newcommand{\VariableTok}[1]{\textcolor[rgb]{0.10,0.09,0.49}{{#1}}}
    \newcommand{\ControlFlowTok}[1]{\textcolor[rgb]{0.00,0.44,0.13}{\textbf{{#1}}}}
    \newcommand{\OperatorTok}[1]{\textcolor[rgb]{0.40,0.40,0.40}{{#1}}}
    \newcommand{\BuiltInTok}[1]{{#1}}
    \newcommand{\ExtensionTok}[1]{{#1}}
    \newcommand{\PreprocessorTok}[1]{\textcolor[rgb]{0.74,0.48,0.00}{{#1}}}
    \newcommand{\AttributeTok}[1]{\textcolor[rgb]{0.49,0.56,0.16}{{#1}}}
    \newcommand{\InformationTok}[1]{\textcolor[rgb]{0.38,0.63,0.69}{\textbf{\textit{{#1}}}}}
    \newcommand{\WarningTok}[1]{\textcolor[rgb]{0.38,0.63,0.69}{\textbf{\textit{{#1}}}}}
    
    
    % Define a nice break command that doesn't care if a line doesn't already
    % exist.
    \def\br{\hspace*{\fill} \\* }
    % Math Jax compatibility definitions
    \def\gt{>}
    \def\lt{<}
    \let\Oldtex\TeX
    \let\Oldlatex\LaTeX
    \renewcommand{\TeX}{\textrm{\Oldtex}}
    \renewcommand{\LaTeX}{\textrm{\Oldlatex}}
    % Document parameters
    % Document title
    \title{Newton\_MP}
    
    
    
    
    
% Pygments definitions
\makeatletter
\def\PY@reset{\let\PY@it=\relax \let\PY@bf=\relax%
    \let\PY@ul=\relax \let\PY@tc=\relax%
    \let\PY@bc=\relax \let\PY@ff=\relax}
\def\PY@tok#1{\csname PY@tok@#1\endcsname}
\def\PY@toks#1+{\ifx\relax#1\empty\else%
    \PY@tok{#1}\expandafter\PY@toks\fi}
\def\PY@do#1{\PY@bc{\PY@tc{\PY@ul{%
    \PY@it{\PY@bf{\PY@ff{#1}}}}}}}
\def\PY#1#2{\PY@reset\PY@toks#1+\relax+\PY@do{#2}}

\expandafter\def\csname PY@tok@w\endcsname{\def\PY@tc##1{\textcolor[rgb]{0.73,0.73,0.73}{##1}}}
\expandafter\def\csname PY@tok@c\endcsname{\let\PY@it=\textit\def\PY@tc##1{\textcolor[rgb]{0.25,0.50,0.50}{##1}}}
\expandafter\def\csname PY@tok@cp\endcsname{\def\PY@tc##1{\textcolor[rgb]{0.74,0.48,0.00}{##1}}}
\expandafter\def\csname PY@tok@k\endcsname{\let\PY@bf=\textbf\def\PY@tc##1{\textcolor[rgb]{0.00,0.50,0.00}{##1}}}
\expandafter\def\csname PY@tok@kp\endcsname{\def\PY@tc##1{\textcolor[rgb]{0.00,0.50,0.00}{##1}}}
\expandafter\def\csname PY@tok@kt\endcsname{\def\PY@tc##1{\textcolor[rgb]{0.69,0.00,0.25}{##1}}}
\expandafter\def\csname PY@tok@o\endcsname{\def\PY@tc##1{\textcolor[rgb]{0.40,0.40,0.40}{##1}}}
\expandafter\def\csname PY@tok@ow\endcsname{\let\PY@bf=\textbf\def\PY@tc##1{\textcolor[rgb]{0.67,0.13,1.00}{##1}}}
\expandafter\def\csname PY@tok@nb\endcsname{\def\PY@tc##1{\textcolor[rgb]{0.00,0.50,0.00}{##1}}}
\expandafter\def\csname PY@tok@nf\endcsname{\def\PY@tc##1{\textcolor[rgb]{0.00,0.00,1.00}{##1}}}
\expandafter\def\csname PY@tok@nc\endcsname{\let\PY@bf=\textbf\def\PY@tc##1{\textcolor[rgb]{0.00,0.00,1.00}{##1}}}
\expandafter\def\csname PY@tok@nn\endcsname{\let\PY@bf=\textbf\def\PY@tc##1{\textcolor[rgb]{0.00,0.00,1.00}{##1}}}
\expandafter\def\csname PY@tok@ne\endcsname{\let\PY@bf=\textbf\def\PY@tc##1{\textcolor[rgb]{0.82,0.25,0.23}{##1}}}
\expandafter\def\csname PY@tok@nv\endcsname{\def\PY@tc##1{\textcolor[rgb]{0.10,0.09,0.49}{##1}}}
\expandafter\def\csname PY@tok@no\endcsname{\def\PY@tc##1{\textcolor[rgb]{0.53,0.00,0.00}{##1}}}
\expandafter\def\csname PY@tok@nl\endcsname{\def\PY@tc##1{\textcolor[rgb]{0.63,0.63,0.00}{##1}}}
\expandafter\def\csname PY@tok@ni\endcsname{\let\PY@bf=\textbf\def\PY@tc##1{\textcolor[rgb]{0.60,0.60,0.60}{##1}}}
\expandafter\def\csname PY@tok@na\endcsname{\def\PY@tc##1{\textcolor[rgb]{0.49,0.56,0.16}{##1}}}
\expandafter\def\csname PY@tok@nt\endcsname{\let\PY@bf=\textbf\def\PY@tc##1{\textcolor[rgb]{0.00,0.50,0.00}{##1}}}
\expandafter\def\csname PY@tok@nd\endcsname{\def\PY@tc##1{\textcolor[rgb]{0.67,0.13,1.00}{##1}}}
\expandafter\def\csname PY@tok@s\endcsname{\def\PY@tc##1{\textcolor[rgb]{0.73,0.13,0.13}{##1}}}
\expandafter\def\csname PY@tok@sd\endcsname{\let\PY@it=\textit\def\PY@tc##1{\textcolor[rgb]{0.73,0.13,0.13}{##1}}}
\expandafter\def\csname PY@tok@si\endcsname{\let\PY@bf=\textbf\def\PY@tc##1{\textcolor[rgb]{0.73,0.40,0.53}{##1}}}
\expandafter\def\csname PY@tok@se\endcsname{\let\PY@bf=\textbf\def\PY@tc##1{\textcolor[rgb]{0.73,0.40,0.13}{##1}}}
\expandafter\def\csname PY@tok@sr\endcsname{\def\PY@tc##1{\textcolor[rgb]{0.73,0.40,0.53}{##1}}}
\expandafter\def\csname PY@tok@ss\endcsname{\def\PY@tc##1{\textcolor[rgb]{0.10,0.09,0.49}{##1}}}
\expandafter\def\csname PY@tok@sx\endcsname{\def\PY@tc##1{\textcolor[rgb]{0.00,0.50,0.00}{##1}}}
\expandafter\def\csname PY@tok@m\endcsname{\def\PY@tc##1{\textcolor[rgb]{0.40,0.40,0.40}{##1}}}
\expandafter\def\csname PY@tok@gh\endcsname{\let\PY@bf=\textbf\def\PY@tc##1{\textcolor[rgb]{0.00,0.00,0.50}{##1}}}
\expandafter\def\csname PY@tok@gu\endcsname{\let\PY@bf=\textbf\def\PY@tc##1{\textcolor[rgb]{0.50,0.00,0.50}{##1}}}
\expandafter\def\csname PY@tok@gd\endcsname{\def\PY@tc##1{\textcolor[rgb]{0.63,0.00,0.00}{##1}}}
\expandafter\def\csname PY@tok@gi\endcsname{\def\PY@tc##1{\textcolor[rgb]{0.00,0.63,0.00}{##1}}}
\expandafter\def\csname PY@tok@gr\endcsname{\def\PY@tc##1{\textcolor[rgb]{1.00,0.00,0.00}{##1}}}
\expandafter\def\csname PY@tok@ge\endcsname{\let\PY@it=\textit}
\expandafter\def\csname PY@tok@gs\endcsname{\let\PY@bf=\textbf}
\expandafter\def\csname PY@tok@gp\endcsname{\let\PY@bf=\textbf\def\PY@tc##1{\textcolor[rgb]{0.00,0.00,0.50}{##1}}}
\expandafter\def\csname PY@tok@go\endcsname{\def\PY@tc##1{\textcolor[rgb]{0.53,0.53,0.53}{##1}}}
\expandafter\def\csname PY@tok@gt\endcsname{\def\PY@tc##1{\textcolor[rgb]{0.00,0.27,0.87}{##1}}}
\expandafter\def\csname PY@tok@err\endcsname{\def\PY@bc##1{\setlength{\fboxsep}{0pt}\fcolorbox[rgb]{1.00,0.00,0.00}{1,1,1}{\strut ##1}}}
\expandafter\def\csname PY@tok@kc\endcsname{\let\PY@bf=\textbf\def\PY@tc##1{\textcolor[rgb]{0.00,0.50,0.00}{##1}}}
\expandafter\def\csname PY@tok@kd\endcsname{\let\PY@bf=\textbf\def\PY@tc##1{\textcolor[rgb]{0.00,0.50,0.00}{##1}}}
\expandafter\def\csname PY@tok@kn\endcsname{\let\PY@bf=\textbf\def\PY@tc##1{\textcolor[rgb]{0.00,0.50,0.00}{##1}}}
\expandafter\def\csname PY@tok@kr\endcsname{\let\PY@bf=\textbf\def\PY@tc##1{\textcolor[rgb]{0.00,0.50,0.00}{##1}}}
\expandafter\def\csname PY@tok@bp\endcsname{\def\PY@tc##1{\textcolor[rgb]{0.00,0.50,0.00}{##1}}}
\expandafter\def\csname PY@tok@fm\endcsname{\def\PY@tc##1{\textcolor[rgb]{0.00,0.00,1.00}{##1}}}
\expandafter\def\csname PY@tok@vc\endcsname{\def\PY@tc##1{\textcolor[rgb]{0.10,0.09,0.49}{##1}}}
\expandafter\def\csname PY@tok@vg\endcsname{\def\PY@tc##1{\textcolor[rgb]{0.10,0.09,0.49}{##1}}}
\expandafter\def\csname PY@tok@vi\endcsname{\def\PY@tc##1{\textcolor[rgb]{0.10,0.09,0.49}{##1}}}
\expandafter\def\csname PY@tok@vm\endcsname{\def\PY@tc##1{\textcolor[rgb]{0.10,0.09,0.49}{##1}}}
\expandafter\def\csname PY@tok@sa\endcsname{\def\PY@tc##1{\textcolor[rgb]{0.73,0.13,0.13}{##1}}}
\expandafter\def\csname PY@tok@sb\endcsname{\def\PY@tc##1{\textcolor[rgb]{0.73,0.13,0.13}{##1}}}
\expandafter\def\csname PY@tok@sc\endcsname{\def\PY@tc##1{\textcolor[rgb]{0.73,0.13,0.13}{##1}}}
\expandafter\def\csname PY@tok@dl\endcsname{\def\PY@tc##1{\textcolor[rgb]{0.73,0.13,0.13}{##1}}}
\expandafter\def\csname PY@tok@s2\endcsname{\def\PY@tc##1{\textcolor[rgb]{0.73,0.13,0.13}{##1}}}
\expandafter\def\csname PY@tok@sh\endcsname{\def\PY@tc##1{\textcolor[rgb]{0.73,0.13,0.13}{##1}}}
\expandafter\def\csname PY@tok@s1\endcsname{\def\PY@tc##1{\textcolor[rgb]{0.73,0.13,0.13}{##1}}}
\expandafter\def\csname PY@tok@mb\endcsname{\def\PY@tc##1{\textcolor[rgb]{0.40,0.40,0.40}{##1}}}
\expandafter\def\csname PY@tok@mf\endcsname{\def\PY@tc##1{\textcolor[rgb]{0.40,0.40,0.40}{##1}}}
\expandafter\def\csname PY@tok@mh\endcsname{\def\PY@tc##1{\textcolor[rgb]{0.40,0.40,0.40}{##1}}}
\expandafter\def\csname PY@tok@mi\endcsname{\def\PY@tc##1{\textcolor[rgb]{0.40,0.40,0.40}{##1}}}
\expandafter\def\csname PY@tok@il\endcsname{\def\PY@tc##1{\textcolor[rgb]{0.40,0.40,0.40}{##1}}}
\expandafter\def\csname PY@tok@mo\endcsname{\def\PY@tc##1{\textcolor[rgb]{0.40,0.40,0.40}{##1}}}
\expandafter\def\csname PY@tok@ch\endcsname{\let\PY@it=\textit\def\PY@tc##1{\textcolor[rgb]{0.25,0.50,0.50}{##1}}}
\expandafter\def\csname PY@tok@cm\endcsname{\let\PY@it=\textit\def\PY@tc##1{\textcolor[rgb]{0.25,0.50,0.50}{##1}}}
\expandafter\def\csname PY@tok@cpf\endcsname{\let\PY@it=\textit\def\PY@tc##1{\textcolor[rgb]{0.25,0.50,0.50}{##1}}}
\expandafter\def\csname PY@tok@c1\endcsname{\let\PY@it=\textit\def\PY@tc##1{\textcolor[rgb]{0.25,0.50,0.50}{##1}}}
\expandafter\def\csname PY@tok@cs\endcsname{\let\PY@it=\textit\def\PY@tc##1{\textcolor[rgb]{0.25,0.50,0.50}{##1}}}

\def\PYZbs{\char`\\}
\def\PYZus{\char`\_}
\def\PYZob{\char`\{}
\def\PYZcb{\char`\}}
\def\PYZca{\char`\^}
\def\PYZam{\char`\&}
\def\PYZlt{\char`\<}
\def\PYZgt{\char`\>}
\def\PYZsh{\char`\#}
\def\PYZpc{\char`\%}
\def\PYZdl{\char`\$}
\def\PYZhy{\char`\-}
\def\PYZsq{\char`\'}
\def\PYZdq{\char`\"}
\def\PYZti{\char`\~}
% for compatibility with earlier versions
\def\PYZat{@}
\def\PYZlb{[}
\def\PYZrb{]}
\makeatother


    % For linebreaks inside Verbatim environment from package fancyvrb. 
    \makeatletter
        \newbox\Wrappedcontinuationbox 
        \newbox\Wrappedvisiblespacebox 
        \newcommand*\Wrappedvisiblespace {\textcolor{red}{\textvisiblespace}} 
        \newcommand*\Wrappedcontinuationsymbol {\textcolor{red}{\llap{\tiny$\m@th\hookrightarrow$}}} 
        \newcommand*\Wrappedcontinuationindent {3ex } 
        \newcommand*\Wrappedafterbreak {\kern\Wrappedcontinuationindent\copy\Wrappedcontinuationbox} 
        % Take advantage of the already applied Pygments mark-up to insert 
        % potential linebreaks for TeX processing. 
        %        {, <, #, %, $, ' and ": go to next line. 
        %        _, }, ^, &, >, - and ~: stay at end of broken line. 
        % Use of \textquotesingle for straight quote. 
        \newcommand*\Wrappedbreaksatspecials {% 
            \def\PYGZus{\discretionary{\char`\_}{\Wrappedafterbreak}{\char`\_}}% 
            \def\PYGZob{\discretionary{}{\Wrappedafterbreak\char`\{}{\char`\{}}% 
            \def\PYGZcb{\discretionary{\char`\}}{\Wrappedafterbreak}{\char`\}}}% 
            \def\PYGZca{\discretionary{\char`\^}{\Wrappedafterbreak}{\char`\^}}% 
            \def\PYGZam{\discretionary{\char`\&}{\Wrappedafterbreak}{\char`\&}}% 
            \def\PYGZlt{\discretionary{}{\Wrappedafterbreak\char`\<}{\char`\<}}% 
            \def\PYGZgt{\discretionary{\char`\>}{\Wrappedafterbreak}{\char`\>}}% 
            \def\PYGZsh{\discretionary{}{\Wrappedafterbreak\char`\#}{\char`\#}}% 
            \def\PYGZpc{\discretionary{}{\Wrappedafterbreak\char`\%}{\char`\%}}% 
            \def\PYGZdl{\discretionary{}{\Wrappedafterbreak\char`\$}{\char`\$}}% 
            \def\PYGZhy{\discretionary{\char`\-}{\Wrappedafterbreak}{\char`\-}}% 
            \def\PYGZsq{\discretionary{}{\Wrappedafterbreak\textquotesingle}{\textquotesingle}}% 
            \def\PYGZdq{\discretionary{}{\Wrappedafterbreak\char`\"}{\char`\"}}% 
            \def\PYGZti{\discretionary{\char`\~}{\Wrappedafterbreak}{\char`\~}}% 
        } 
        % Some characters . , ; ? ! / are not pygmentized. 
        % This macro makes them "active" and they will insert potential linebreaks 
        \newcommand*\Wrappedbreaksatpunct {% 
            \lccode`\~`\.\lowercase{\def~}{\discretionary{\hbox{\char`\.}}{\Wrappedafterbreak}{\hbox{\char`\.}}}% 
            \lccode`\~`\,\lowercase{\def~}{\discretionary{\hbox{\char`\,}}{\Wrappedafterbreak}{\hbox{\char`\,}}}% 
            \lccode`\~`\;\lowercase{\def~}{\discretionary{\hbox{\char`\;}}{\Wrappedafterbreak}{\hbox{\char`\;}}}% 
            \lccode`\~`\:\lowercase{\def~}{\discretionary{\hbox{\char`\:}}{\Wrappedafterbreak}{\hbox{\char`\:}}}% 
            \lccode`\~`\?\lowercase{\def~}{\discretionary{\hbox{\char`\?}}{\Wrappedafterbreak}{\hbox{\char`\?}}}% 
            \lccode`\~`\!\lowercase{\def~}{\discretionary{\hbox{\char`\!}}{\Wrappedafterbreak}{\hbox{\char`\!}}}% 
            \lccode`\~`\/\lowercase{\def~}{\discretionary{\hbox{\char`\/}}{\Wrappedafterbreak}{\hbox{\char`\/}}}% 
            \catcode`\.\active
            \catcode`\,\active 
            \catcode`\;\active
            \catcode`\:\active
            \catcode`\?\active
            \catcode`\!\active
            \catcode`\/\active 
            \lccode`\~`\~ 	
        }
    \makeatother

    \let\OriginalVerbatim=\Verbatim
    \makeatletter
    \renewcommand{\Verbatim}[1][1]{%
        %\parskip\z@skip
        \sbox\Wrappedcontinuationbox {\Wrappedcontinuationsymbol}%
        \sbox\Wrappedvisiblespacebox {\FV@SetupFont\Wrappedvisiblespace}%
        \def\FancyVerbFormatLine ##1{\hsize\linewidth
            \vtop{\raggedright\hyphenpenalty\z@\exhyphenpenalty\z@
                \doublehyphendemerits\z@\finalhyphendemerits\z@
                \strut ##1\strut}%
        }%
        % If the linebreak is at a space, the latter will be displayed as visible
        % space at end of first line, and a continuation symbol starts next line.
        % Stretch/shrink are however usually zero for typewriter font.
        \def\FV@Space {%
            \nobreak\hskip\z@ plus\fontdimen3\font minus\fontdimen4\font
            \discretionary{\copy\Wrappedvisiblespacebox}{\Wrappedafterbreak}
            {\kern\fontdimen2\font}%
        }%
        
        % Allow breaks at special characters using \PYG... macros.
        \Wrappedbreaksatspecials
        % Breaks at punctuation characters . , ; ? ! and / need catcode=\active 	
        \OriginalVerbatim[#1,codes*=\Wrappedbreaksatpunct]%
    }
    \makeatother

    % Exact colors from NB
    \definecolor{incolor}{HTML}{303F9F}
    \definecolor{outcolor}{HTML}{D84315}
    \definecolor{cellborder}{HTML}{CFCFCF}
    \definecolor{cellbackground}{HTML}{F7F7F7}
    
    % prompt
    \newcommand{\prompt}[4]{
        \llap{{\color{#2}[#3]: #4}}\vspace{-1.25em}
    }
    

    
    % Prevent overflowing lines due to hard-to-break entities
    \sloppy 
    % Setup hyperref package
    \hypersetup{
      breaklinks=true,  % so long urls are correctly broken across lines
      colorlinks=true,
      urlcolor=urlcolor,
      linkcolor=linkcolor,
      citecolor=citecolor,
      }
    % Slightly bigger margins than the latex defaults
    
    \geometry{verbose,tmargin=1in,bmargin=1in,lmargin=1in,rmargin=1in}
    
    

    \begin{document}
    
    
    \maketitle
    
    

    

\newcommand{\calf}{{\cal F}}
\newcommand{\dnu}{d \nu}
\newcommand{\mf}{{\bf F}}
\newcommand{\vu}{{\bf u}}
\newcommand{\ve}{{\bf e}}
\newcommand{\ml}{{\bf L}}
\newcommand{\mg}{{\bf G}}
\newcommand{\mi}{{\bf I}}
\newcommand{\diag}{\mbox{diag}}
%\newcommand{\begeq}{{\begin{equation}}}
%\newcommand{\endeq}{{\end{equation}}}


\hypertarget{newtons-method-in-multiple-precision-c.-t.-kelley}{%
\section{Newton's Method in Multiple Precision: C. T.
Kelley}\label{newtons-method-in-multiple-precision-c.-t.-kelley}}

This document has two purposes.

\begin{enumerate}
\def\labelenumi{\arabic{enumi}.}
\item
  To document the SIREV paper \cite{ctk:sirev19}
\item
  To show the RTG people how to do some non-trivial work in Julia and
  put some LaTeX in their notebooks.
\end{enumerate}

As an example we will solve the Chandrasekhar H-equation \cite{chand}.
This equation, which we describe in detail in the Example section, has a
fast \(O(N \log(N))\) function evaluation, a Jacobian evaluation that is
\(O(N^2)\) work analytically and \(O(N^2 \log(N))\) with a finite
difference. This means that most of the work, if you do things right, is
in the LU factorization of the Jacobian.

The difference between double, single, and half precision will be clear
in the results from the examples.

    \hypertarget{the-chandresekhar-h-equation}{%
\subsection{The Chandresekhar
H-Equation}\label{the-chandresekhar-h-equation}}

The example is the mid-point rule discretization of the Chandrsekhar
H-equation \cite{chand}.

\begin{equation}
    ({\calf})(\mu) = H(\mu) -
\left(
1 - \frac{c}{2} \int_0^1 \frac{\mu H(\mu)}{\mu + \nu} \dnu
\right)^{-1} = 0.
    \end{equation}

The nonlinear operator \(\calf\) is defined on \(C[0,1]\), the space of
continuous functions on \([0,1]\).

The equation has a well-understood dependence on the parameter \(c\)
\cite{twm68}, \cite{ctk:n1}. The equation has unique solutions at
\(c=0\) and \(c=1\) and two solutions for \(0 < c < 1\). There is a
simple fold singularity \cite{herb} at \(c=1\). Only one \cite{chand},
\cite{busb} of the two solutions for \(0 < c < 1\) is of physical
interest and that is the one easiest to find numerically. One must do a
continuation computation to find the other one.

The structure of the singularity is preserved if one discretizes the
integral with any rule that integrates constants exactly. For the
purposes of this paper the composite midpoint rule will suffice. The
\(N\)-point composite midpoint rule is \begin{equation}
\int_0^1 f(\nu) \dnu \approx \frac{1}{N} \sum_{j=1}^N f(\nu_j)
\end{equation} where \(\nu_j = (j - 1/2)/N\) for \(1 \le j \le N\). This
rule is second-order accurate for sufficiently smooth functions \(f\).
The solution of the integral equation is, however, not smooth enough.
\(H'(\mu)\) has a logarithmic singularity at \(\mu=0\).

The discrete problem is

\begin{equation}
\mf(\vu)_i \equiv
u_i - \left(
1  - \frac{c}{2N} \sum_{j=1}^N \frac{u_j \mu_i}{\mu_j + \mu_i}
\right)^{-1}
=0.
\end{equation}

One can simplify the approximate integral operator 
expose some useful structure. Since

\begin{equation}
\frac{c}{2N} \sum_{j=1}^N \frac{u_j \mu_i}{\mu_j + \mu_i}
= \frac{c (i - 1/2) }{2N} \sum_{j=1}^N \frac{u_j}{i+j -1}.
\end{equation}

hence the approximate integral operator is the product of a diagonal
matrix and a Hankel matrix and one can use an FFT to evaluate that
operator with \(O(N \log(N))\) work \cite{golub}.

We can express the approximation of the integral operator in matrix form
\begin{equation}
\ml(\vu)_i = \frac{c (i - 1/2) }{2N} \sum_{j=1}^N \frac{u_j}{i+j -1}
\end{equation} and compute the Jacobian analytically as \begin{equation}
\mf'(\vu) = \mi - \diag(\mg(\vu))^2 \ml
\end{equation} where \begin{equation}
\mg(\vu)_i = \left(
1  - \frac{c}{2N} \sum_{j=1}^N \frac{u_j \mu_i}{\mu_j + \mu_i}
\right)^{-1}.
\end{equation} Hence the data for the Jacobian is already available
after one computes \(\mf(\vu) = \vu - \mg(\vu)\) and the Jacobian can be
computed with \(O(N^2)\) work. We do that in this example and therefore
the only \(O(N^3)\) part of the solve is the matrix factorization.

One could also approximate the Jacobian with forward differences. In
this case one approximates the \(j\)th column \(\mf'(\vu)_j\) of the
Jacobian with \begin{equation}
\frac{\mf(\vu + h {\tilde \ve}_j) - \mf(\vu)}{h}
\end{equation} where \({\tilde \ve}_j\) is a unit vector in the \(j\)th
coordinate direction and \(h\) is a suitable difference increment. If
one computes \(\mf\) in double precision with unit roundoff \(u_d\),
then \(h =O(\| \vu \| \sqrt{u_d})\) is a reasonable choice
\cite{ctk:roots}. Then the error in the Jacobian is
\(O(\sqrt{u_d}) = O(u_s)\) where \(u_s\) is unit roundoff in single
precision. The cost of a finite difference Jacobian in this example is
\(O(N^2 \log(N))\) work.

The analysis in \cite{ctk:sirev19} suggests that there is no significant
difference in the nonlinear iteration from either the choice of analytic
or finite difference Jacobians or the choice of single or double
precision for the linear solver. This notebook has the data used in that
paper to support that assertion. You will be able to duplicate the
results and play with the codes.

Half precision is another story and we have those codes for you, too.

    \hypertarget{setting-up}{%
\subsection{Setting up}\label{setting-up}}

You need to install these packages:

\begin{itemize}
\tightlist
\item
  PyPlot
\item
  LinearAlgebra
\item
  JLD2
\item
  Printf
\item
  FFTW
\item
  IJulia (You must have done this already or you would not be looking at
  this notebook.)
\item
  AbstractFFTs
\end{itemize}

The directory is a Julia project. So all you should need to do to get
going is to

\begin{enumerate}
\def\labelenumi{\arabic{enumi}.}
\tightlist
\item
  Put the directory in your LOAD\_PATH. The way to do this is to type
\end{enumerate}

\begin{Shaded}
\begin{Highlighting}[]
\NormalTok{push!(LOAD_PATH,}\StringTok{"/Users/yourid/whereyouputit/MPResults2019"}\NormalTok{)}
\end{Highlighting}
\end{Shaded}

at the Julia prompt in the REPL or in a notebook code windown.

\begin{enumerate}
\def\labelenumi{\arabic{enumi}.}
\setcounter{enumi}{1}
\tightlist
\item
  Now load the modules with
\end{enumerate}

\begin{Shaded}
\begin{Highlighting}[]
\NormalTok{using MPResults}\FloatTok{2019}
\end{Highlighting}
\end{Shaded}

\begin{enumerate}
\def\labelenumi{\arabic{enumi}.}
\setcounter{enumi}{2}
\tightlist
\item
  Then you can do a simple solve and test that you did it right by
  typing
\end{enumerate}

\begin{Shaded}
\begin{Highlighting}[]
\NormalTok{hout=heqtest()}
\end{Highlighting}
\end{Shaded}

which I will do now. Make sure \textbf{MPResults2019} is in your
LOAD\_PATH! If you forget to do the push! command, strange things may
happen.

    \begin{tcolorbox}[breakable, size=fbox, boxrule=1pt, pad at break*=1mm,colback=cellbackground, colframe=cellborder]
\prompt{In}{incolor}{8}{\hspace{4pt}}
\begin{Verbatim}[commandchars=\\\{\}]
\PY{n}{push!}\PY{p}{(}\PY{n+nb}{LOAD\PYZus{}PATH}\PY{p}{,}\PY{l+s}{\PYZdq{}}\PY{l+s}{/}\PY{l+s}{U}\PY{l+s}{s}\PY{l+s}{e}\PY{l+s}{r}\PY{l+s}{s}\PY{l+s}{/}\PY{l+s}{c}\PY{l+s}{t}\PY{l+s}{k}\PY{l+s}{/}\PY{l+s}{D}\PY{l+s}{r}\PY{l+s}{o}\PY{l+s}{p}\PY{l+s}{b}\PY{l+s}{o}\PY{l+s}{x}\PY{l+s}{/}\PY{l+s}{J}\PY{l+s}{u}\PY{l+s}{l}\PY{l+s}{i}\PY{l+s}{a}\PY{l+s}{/}\PY{l+s}{M}\PY{l+s}{P}\PY{l+s}{R}\PY{l+s}{e}\PY{l+s}{s}\PY{l+s}{u}\PY{l+s}{l}\PY{l+s}{t}\PY{l+s}{s}\PY{l+s}{2}\PY{l+s}{0}\PY{l+s}{1}\PY{l+s}{9}\PY{l+s}{\PYZdq{}}\PY{p}{)}
\PY{k}{using} \PY{n}{MPResults2019}
\end{Verbatim}
\end{tcolorbox}

    \begin{tcolorbox}[breakable, size=fbox, boxrule=1pt, pad at break*=1mm,colback=cellbackground, colframe=cellborder]
\prompt{In}{incolor}{9}{\hspace{4pt}}
\begin{Verbatim}[commandchars=\\\{\}]
\PY{n}{heqtest}\PY{p}{(}\PY{p}{)}
\end{Verbatim}
\end{tcolorbox}

    \begin{Verbatim}[commandchars=\\\{\}]
0.00e+00      1.00000e+00
5.00e-02      1.04424e+00
1.00e-01      1.07236e+00
1.50e-01      1.09470e+00
2.00e-01      1.11346e+00
2.50e-01      1.12965e+00
3.00e-01      1.14389e+00
3.50e-01      1.15657e+00
4.00e-01      1.16797e+00
4.50e-01      1.17830e+00
5.00e-01      1.18773e+00
5.50e-01      1.19638e+00
6.00e-01      1.20435e+00
6.50e-01      1.21172e+00
7.00e-01      1.21856e+00
7.50e-01      1.22493e+00
8.00e-01      1.23088e+00
8.50e-01      1.23646e+00
9.00e-01      1.24169e+00
9.50e-01      1.24662e+00
1.00e+00      1.25126e+00



[1.54457e+00, 7.94167e-03, 1.55209e-07, 2.67377e-15, 1.53837e-15, 1.35064e-15,
1.52226e-15, 1.42178e-15, 1.47288e-15, 1.42178e-15, 1.33227e-15]
\end{Verbatim}

            \begin{tcolorbox}[breakable, boxrule=.5pt, size=fbox, pad at break*=1mm, opacityfill=0]
\prompt{Out}{outcolor}{9}{\hspace{3.5pt}}
\begin{Verbatim}[commandchars=\\\{\}]
(exactout = (solution = [1.00707e+00; 1.01754e+00; … ; 1.24989e+00;
1.25081e+00], ithist = [1.54457e+00, 7.94167e-03, 1.55209e-07, 2.67377e-15,
1.53837e-15, 1.35064e-15, 1.52226e-15, 1.42178e-15, 1.47288e-15, 1.42178e-15,
1.33227e-15]), fdout = (solution = [1.00707e+00; 1.01754e+00; … ; 1.24989e+00;
1.25081e+00], ithist = [1.54457e+00, 7.94166e-03, 1.55239e-07, 2.24254e-15,
1.58572e-15, 1.53837e-15, 1.47288e-15, 1.53837e-15, 1.67640e-15, 1.57009e-15,
1.23629e-15]))
\end{Verbatim}
\end{tcolorbox}
        
    heqtest.jl, calls the solver and harvests some iteration statistics. The
two columns of numbers are the reults from \cite{chand} (page 125). The
iteration statistics are from KNL, the solver.

    \hypertarget{knl}{%
\subsection{KNL}\label{knl}}

The solver is \emph{knl.jl} version .01. Keep in mind that nothing with
a version number with a negative exponent field is likely to be very
good. knl.jl is included when you run the MPResults2019.jl module, which
you do automatically when you type \textbf{using MPResults2019}

knl.jl is a simple implemention of Newton's method (see \cite{ctk:roots}
and \cite{ctk:newton} ) using an LU factorization of the Jacobian to
compute the Newton step with no line search or globalization. The code
evaluates and factors the Jacobian at every nonlinear iteration.

Compare this with the scalar code with talked about last time.

    \hypertarget{using-knl.jl}{%
\subsection{Using knl.jl}\label{using-knl.jl}}

At the level of this notebook, it's pretty simple. Remember that Julia
hates to allocate mememory. So your function and Jacobian evaluation
routines should expect the calling function to \textbf{preallocate} the
storage for both the function and Jacobian. Your functions will then use
\textbf{.=} to put the function and Jacobian where they are supposed to
be.

It's worthwhile to look at the help screen.

    It's worth asking for help with knl \ldots{}.

    \begin{tcolorbox}[breakable, size=fbox, boxrule=1pt, pad at break*=1mm,colback=cellbackground, colframe=cellborder]
\prompt{In}{incolor}{10}{\hspace{4pt}}
\begin{Verbatim}[commandchars=\\\{\}]
\PY{o}{?}\PY{n}{knl}
\end{Verbatim}
\end{tcolorbox}

    \begin{Verbatim}[commandchars=\\\{\}]
search: \textbf{k}\textbf{n}\textbf{l}
plot\textbf{k}\textbf{n}\textbf{l}
install\textbf{k}er\textbf{n}e\textbf{l}

\end{Verbatim}
 
            
\prompt{Out}{outcolor}{10}{}
    
    \begin{verbatim}
knl(x, FS, FPS, F!, J!=diffjac!; rtol=1.e-6, atol=1.e-12, 
        maxit=20, dx=1.e-6, pdata=nothing)
\end{verbatim}
This is Version .01. Nothing with a version number having a negative exponent field can be trusted.

Nonlinear solvers from my books in Julia. This version has no globalization, no quasi-Newton methods, and no Newton-Krylov. The mission here is to  duplicate the mixed precision results in my SIREV-ED submission.

\section{Inputs:}
\begin{itemize}
\item x: initial iterate


\item FS: Preallcoated storage for function. It is an N x 1 column vector


\item FPS: preallcoated storage for Jacobian. It is an N x N matrix


\item F!: function evaluation, the ! indicates that F! overwrites FS, your   preallocated storage for the function.


\item J!: Jacobian evaluation, the ! indicates that J! overwrites FPS, your   preallocated storage for the Jacobian. If you leave this out the   default is a finite difference Jacobian.

\end{itemize}
\rule{\textwidth}{1pt}
\section{keyword arguments (kwargs):}
\begin{itemize}
\item rtol and atol: relative and absolute error tolerances


\item maxit: limit on nonlinear iterations


\item dx: difference increment in finite-difference derivatives     h=dx*norm(x)+1.e-8


\item pdata: precomputed data for the function/Jacobian.       Things will go better if you use this rather than hide the data       in global variables within the module for your function/Jacobian

\end{itemize}
\rule{\textwidth}{1pt}
\section{Using knl.jl}
Here are the rules as of June 6, 2019

F! is the nonlinear residual.  J! is the Jacobian evaluation.

Put these things in a module and cook it up so that

\begin{itemize}
\item[1. ] You allocate storage for the function and Jacobian in advance  –> in the calling program <– NOT in FS and FPS

\end{itemize}
FV=F!(FV,x) returns FV=F(x)

FP=J!(FV,FP,x) returns FP=F'(x);      (FV,FP, x) must be the argument list, even if FP does not need FV.     One reason for this is that the finite-difference Jacobian     does and that is the default in the solver.

In the future J! will also be a matrix-vector product and FPS will be the PREALLOCATED (!!) storage for the GMRES(m) Krylov vectors.

Lemme tell ya 'bout precision. I designed this code for full precision functions and linear algebra in any precision you want. You can decleare FPS as Float64, Float32, or Float16 and knl will do the right thing if  YOU do not destroy the declaration in your J! function. I'm amazed  that this works so easily. 

If the Jacobian is reasonably well conditioned, I can see no reason to do linear algebra in double precision

Don't try to evaluate function and Jacobian all at once because  that will cost you a extra function evaluation everytime the line search kicks in.

\begin{itemize}
\item[2. ] Any precomputed data for functions, Jacobians, matrix-vector products may live in global variables within the module. Don't do that if you can avoid it. Use pdata instead.

\end{itemize}


    

    \hypertarget{how-knl.jl-controls-the-precision-of-the-jacobian}{%
\subsection{How knl.jl controls the precision of the
Jacobian}\label{how-knl.jl-controls-the-precision-of-the-jacobian}}

You can control the precision of the Jacobian by simply allocating FPS
in your favorite precision. So if I have a problem with N=256 unknows I
will decalare FP as zeros(N,1) and may delcare FPS as zeros(N,N) or
Float32.(zeros(N,N)).

Note the \textbf{.} between Float32 and the paren. This, as is standard
Julia practice, applies the conversion to Float32 to everyelement in the
array. If you forget the \textbf{.} Julia will complain.

    \hypertarget{the-results-in-the-paper}{%
\section{The results in the paper}\label{the-results-in-the-paper}}

The paper has plots for double, single, and half precsion computations
for c=.5, .99, and 1.0. The half precision results take a very long time
to get. On my computer (2019 iMac; 8 cores, 64GB of memoroy) the half
precision compute time was over two weeks. Kids don't try this at home.

The data for the paper are in the cleverly named directory
\textbf{Data\_From\_Paper}

\textbf{cd to the directory MPResults2019} and \textbf{from that
directory} run

\begin{Shaded}
\begin{Highlighting}[]
\NormalTok{data_harvest(}\StringTok{"Data_From_Paper/MP_Data_"}\NormalTok{) }
\end{Highlighting}
\end{Shaded}

at the julia prompt you will generate all the tables and plots in the
paper.

If you have the time and patience you can also generate the data with
data\_populate.jl. This create three directories named
Mixed\_Precision\_c=? and you can see for yourself. Look at that file to
see opportunities to edit the time-cosumng jobs out. If you eliminate
the half precision work and the larger dimensions, the code will run in
a short time.

Here is a simple example of using data\_populate and the plotter code
plot\_knl.jl. I'm only using the 1024, 2048 and 4096 point grids. The
plot in the paper uses more levels. This is part of Figure 1 in the
paper.

Look at the source to \textbf{data\_populate.jl} and \textbf{plotknl.jl}
and you'll see how I did this. These codes only mange files, plots, and
tables. There is nothing really exciting here. You don't need to know
much Julia to understand this, but you do need to know something and I
can't help with that.

    \begin{tcolorbox}[breakable, size=fbox, boxrule=1pt, pad at break*=1mm,colback=cellbackground, colframe=cellborder]
\prompt{In}{incolor}{11}{\hspace{4pt}}
\begin{Verbatim}[commandchars=\\\{\}]
\PY{n}{Home4mp}\PY{o}{=}\PY{l+s}{\PYZdq{}}\PY{l+s}{/}\PY{l+s}{U}\PY{l+s}{s}\PY{l+s}{e}\PY{l+s}{r}\PY{l+s}{s}\PY{l+s}{/}\PY{l+s}{c}\PY{l+s}{t}\PY{l+s}{k}\PY{l+s}{/}\PY{l+s}{D}\PY{l+s}{r}\PY{l+s}{o}\PY{l+s}{p}\PY{l+s}{b}\PY{l+s}{o}\PY{l+s}{x}\PY{l+s}{/}\PY{l+s}{J}\PY{l+s}{u}\PY{l+s}{l}\PY{l+s}{i}\PY{l+s}{a}\PY{l+s}{/}\PY{l+s}{M}\PY{l+s}{P}\PY{l+s}{R}\PY{l+s}{e}\PY{l+s}{s}\PY{l+s}{u}\PY{l+s}{l}\PY{l+s}{t}\PY{l+s}{s}\PY{l+s}{2}\PY{l+s}{0}\PY{l+s}{1}\PY{l+s}{9}\PY{l+s}{\PYZdq{}}
\PY{n}{cd}\PY{p}{(}\PY{n}{Home4mp}\PY{p}{)}
\PY{n}{pwd}\PY{p}{(}\PY{p}{)}
\end{Verbatim}
\end{tcolorbox}

            \begin{tcolorbox}[breakable, boxrule=.5pt, size=fbox, pad at break*=1mm, opacityfill=0]
\prompt{Out}{outcolor}{11}{\hspace{3.5pt}}
\begin{Verbatim}[commandchars=\\\{\}]
"/Users/ctk/Dropbox/Julia/MPResults2019"
\end{Verbatim}
\end{tcolorbox}
        
    \begin{tcolorbox}[breakable, size=fbox, boxrule=1pt, pad at break*=1mm,colback=cellbackground, colframe=cellborder]
\prompt{In}{incolor}{12}{\hspace{4pt}}
\begin{Verbatim}[commandchars=\\\{\}]
\PY{n}{cd}\PY{p}{(}\PY{n}{Home4mp}\PY{p}{)}
\PY{k}{using} \PY{n}{PyPlot}
\PY{n}{data\PYZus{}populate}\PY{p}{(}\PY{o}{.}\PY{l+m+mi}{5}\PY{p}{;}\PY{n}{half}\PY{o}{=}\PY{l+s}{\PYZdq{}}\PY{l+s}{n}\PY{l+s}{o}\PY{l+s}{\PYZdq{}}\PY{p}{,}\PY{n}{level}\PY{o}{=}\PY{l+m+mi}{3}\PY{p}{)}
\end{Verbatim}
\end{tcolorbox}

    \begin{tcolorbox}[breakable, size=fbox, boxrule=1pt, pad at break*=1mm,colback=cellbackground, colframe=cellborder]
\prompt{In}{incolor}{13}{\hspace{4pt}}
\begin{Verbatim}[commandchars=\\\{\}]
\PY{n}{cd}\PY{p}{(}\PY{n}{Home4mp}\PY{p}{)}
\PY{n}{cd}\PY{p}{(}\PY{l+s}{\PYZdq{}}\PY{l+s}{M}\PY{l+s}{i}\PY{l+s}{x}\PY{l+s}{e}\PY{l+s}{d}\PY{l+s}{\PYZus{}}\PY{l+s}{P}\PY{l+s}{r}\PY{l+s}{e}\PY{l+s}{c}\PY{l+s}{i}\PY{l+s}{s}\PY{l+s}{i}\PY{l+s}{o}\PY{l+s}{n}\PY{l+s}{\PYZus{}}\PY{l+s}{c}\PY{l+s}{=}\PY{l+s}{5}\PY{l+s}{\PYZdq{}}\PY{p}{)}
\PY{n}{figtitle}\PY{o}{=}\PY{l+s}{\PYZdq{}}\PY{l+s}{F}\PY{l+s}{i}\PY{l+s}{g}\PY{l+s}{u}\PY{l+s}{r}\PY{l+s}{e}\PY{l+s}{ }\PY{l+s}{1}\PY{l+s}{\PYZdq{}}
\PY{n}{plotknl}\PY{p}{(}\PY{l+s}{\PYZdq{}}\PY{l+s}{n}\PY{l+s}{o}\PY{l+s}{\PYZdq{}}\PY{p}{,}\PY{o}{.}\PY{l+m+mi}{5}\PY{p}{,}\PY{l+m+mi}{10}\PY{p}{,}\PY{l+m+mi}{3}\PY{p}{;}\PY{n}{bigtitle}\PY{o}{=}\PY{n}{figtitle}\PY{p}{)}
\end{Verbatim}
\end{tcolorbox}

    \begin{center}
    \adjustimage{max size={0.9\linewidth}{0.9\paperheight}}{output_14_0.png}
    \end{center}
    { \hspace*{\fill} \\}
    
    \textbf{data\_populate} will create directories for your data if they
are not already there. Run \textbf{plot\_knl} from one of those
directories and it makes the plots. As you can see from the one above,
it's hard to tell the difference between double and single precision
linear algebra and analytic or finite difference Jacobians. You knew
that.


    % Add a bibliography block to the postdoc
    
\bibliographystyle{siam}
\bibliography{newtonmp}

    
    \end{document}
